% Options for packages loaded elsewhere
% Options for packages loaded elsewhere
\PassOptionsToPackage{unicode}{hyperref}
\PassOptionsToPackage{hyphens}{url}
\PassOptionsToPackage{dvipsnames,svgnames,x11names}{xcolor}
%
\documentclass[
  letterpaper,
  DIV=11,
  numbers=noendperiod]{scrartcl}
\usepackage{xcolor}
\usepackage{amsmath,amssymb}
\setcounter{secnumdepth}{-\maxdimen} % remove section numbering
\usepackage{iftex}
\ifPDFTeX
  \usepackage[T1]{fontenc}
  \usepackage[utf8]{inputenc}
  \usepackage{textcomp} % provide euro and other symbols
\else % if luatex or xetex
  \usepackage{unicode-math} % this also loads fontspec
  \defaultfontfeatures{Scale=MatchLowercase}
  \defaultfontfeatures[\rmfamily]{Ligatures=TeX,Scale=1}
\fi
\usepackage{lmodern}
\ifPDFTeX\else
  % xetex/luatex font selection
\fi
% Use upquote if available, for straight quotes in verbatim environments
\IfFileExists{upquote.sty}{\usepackage{upquote}}{}
\IfFileExists{microtype.sty}{% use microtype if available
  \usepackage[]{microtype}
  \UseMicrotypeSet[protrusion]{basicmath} % disable protrusion for tt fonts
}{}
\makeatletter
\@ifundefined{KOMAClassName}{% if non-KOMA class
  \IfFileExists{parskip.sty}{%
    \usepackage{parskip}
  }{% else
    \setlength{\parindent}{0pt}
    \setlength{\parskip}{6pt plus 2pt minus 1pt}}
}{% if KOMA class
  \KOMAoptions{parskip=half}}
\makeatother
% Make \paragraph and \subparagraph free-standing
\makeatletter
\ifx\paragraph\undefined\else
  \let\oldparagraph\paragraph
  \renewcommand{\paragraph}{
    \@ifstar
      \xxxParagraphStar
      \xxxParagraphNoStar
  }
  \newcommand{\xxxParagraphStar}[1]{\oldparagraph*{#1}\mbox{}}
  \newcommand{\xxxParagraphNoStar}[1]{\oldparagraph{#1}\mbox{}}
\fi
\ifx\subparagraph\undefined\else
  \let\oldsubparagraph\subparagraph
  \renewcommand{\subparagraph}{
    \@ifstar
      \xxxSubParagraphStar
      \xxxSubParagraphNoStar
  }
  \newcommand{\xxxSubParagraphStar}[1]{\oldsubparagraph*{#1}\mbox{}}
  \newcommand{\xxxSubParagraphNoStar}[1]{\oldsubparagraph{#1}\mbox{}}
\fi
\makeatother


\usepackage{longtable,booktabs,array}
\usepackage{calc} % for calculating minipage widths
% Correct order of tables after \paragraph or \subparagraph
\usepackage{etoolbox}
\makeatletter
\patchcmd\longtable{\par}{\if@noskipsec\mbox{}\fi\par}{}{}
\makeatother
% Allow footnotes in longtable head/foot
\IfFileExists{footnotehyper.sty}{\usepackage{footnotehyper}}{\usepackage{footnote}}
\makesavenoteenv{longtable}
\usepackage{graphicx}
\makeatletter
\newsavebox\pandoc@box
\newcommand*\pandocbounded[1]{% scales image to fit in text height/width
  \sbox\pandoc@box{#1}%
  \Gscale@div\@tempa{\textheight}{\dimexpr\ht\pandoc@box+\dp\pandoc@box\relax}%
  \Gscale@div\@tempb{\linewidth}{\wd\pandoc@box}%
  \ifdim\@tempb\p@<\@tempa\p@\let\@tempa\@tempb\fi% select the smaller of both
  \ifdim\@tempa\p@<\p@\scalebox{\@tempa}{\usebox\pandoc@box}%
  \else\usebox{\pandoc@box}%
  \fi%
}
% Set default figure placement to htbp
\def\fps@figure{htbp}
\makeatother





\setlength{\emergencystretch}{3em} % prevent overfull lines

\providecommand{\tightlist}{%
  \setlength{\itemsep}{0pt}\setlength{\parskip}{0pt}}



 
\usepackage[]{natbib}
\bibliographystyle{plainnat}


\usepackage{sidecap}
\KOMAoption{captions}{tableheading}
\makeatletter
\@ifpackageloaded{caption}{}{\usepackage{caption}}
\AtBeginDocument{%
\ifdefined\contentsname
  \renewcommand*\contentsname{Table of contents}
\else
  \newcommand\contentsname{Table of contents}
\fi
\ifdefined\listfigurename
  \renewcommand*\listfigurename{List of Figures}
\else
  \newcommand\listfigurename{List of Figures}
\fi
\ifdefined\listtablename
  \renewcommand*\listtablename{List of Tables}
\else
  \newcommand\listtablename{List of Tables}
\fi
\ifdefined\figurename
  \renewcommand*\figurename{Figure}
\else
  \newcommand\figurename{Figure}
\fi
\ifdefined\tablename
  \renewcommand*\tablename{Table}
\else
  \newcommand\tablename{Table}
\fi
}
\@ifpackageloaded{float}{}{\usepackage{float}}
\floatstyle{ruled}
\@ifundefined{c@chapter}{\newfloat{codelisting}{h}{lop}}{\newfloat{codelisting}{h}{lop}[chapter]}
\floatname{codelisting}{Listing}
\newcommand*\listoflistings{\listof{codelisting}{List of Listings}}
\makeatother
\makeatletter
\makeatother
\makeatletter
\@ifpackageloaded{caption}{}{\usepackage{caption}}
\@ifpackageloaded{subcaption}{}{\usepackage{subcaption}}
\makeatother
\usepackage{bookmark}
\IfFileExists{xurl.sty}{\usepackage{xurl}}{} % add URL line breaks if available
\urlstyle{same}
\hypersetup{
  pdftitle={Relativistic electrons in the radiation belt and Current sheet in the solar wind},
  colorlinks=true,
  linkcolor={blue},
  filecolor={Maroon},
  citecolor={Blue},
  urlcolor={Blue},
  pdfcreator={LaTeX via pandoc}}


\title{Relativistic electrons in the radiation belt and Current sheet in the solar wind}
\usepackage{etoolbox}
\makeatletter
\providecommand{\subtitle}[1]{% add subtitle to \maketitle
  \apptocmd{\@title}{\par {\large #1 \par}}{}{}
}
\makeatother
\subtitle{First Oral Exam}
\author{}
\date{}
\begin{document}
\maketitle

\vspace{-20truemm}


Graduate Student: Zijin Zhang
Supervisor: Vassilis Angelopoulos
Committee Members: Marco Velli, Hao Cao, Anton Artemyev

\subsection{Part 1: Relativistic electron flux decay and recovery: relative roles of EMIC waves, chorus waves, and electron injections}\label{part-1-relativistic-electron-flux-decay-and-recovery-relative-roles-of-emic-waves-chorus-waves-and-electron-injections}

\subsection{Conclusion}\label{conclusion}

\subsubsection{Preliminary Results}\label{preliminary-results}

\begin{itemize}
\item
  We examined a particular event on 17 April 2021 characterized by a series of strong electron and ion injections, significant electron precipitation driven by EMIC and chorus waves, and electron acceleration mainly attributable to chorus waves.
\item
  This case study is unique in the sense that strong EMIC and chorus wave-driven electron losses do not necessarily correspond to a simultaneous decrease of trapped electron fluxes. Sufficiently strong injections and chorus wave-driven electron acceleration in the presence of a sufficiently steep negative electron energy PSD gradient can balance such wave-driven losses.
\end{itemize}

\subsubsection{Future Work}\label{future-work}

\begin{itemize}
\item
  Statistically study the conditions that lead to relativistic electron flux decay and recovery.
\item
  Understand the physical mechanisms that lead to the observed electron flux: strong diffusion or non-linear wave-particle interactions.
\end{itemize}

\subsection{Part 2: Current sheet in the solar wind: JUNO and PSP Observations}\label{part-2-current-sheet-in-the-solar-wind-juno-and-psp-observations}

\subsubsection{Preliminary Results}\label{preliminary-results-1}

\begin{itemize}
\item
  The normalized occurrence rate decreases with radial distance from the Sun, following a 1/𝑟 relationship in the outer heliosphere.
\item
  Normalized thickness and current density of discontinuities remain constant with radial distance (negligible change compared to their spread)

  \begin{itemize}
  \tightlist
  \item
    Thickness =\textgreater{} ion inertial length
  \end{itemize}
\item
  Current density =\textgreater{} Alfven velocity (current density)
\item
  Better alignment period has a slightly better agreement of the properties of the discontinuities (normalized thickness, current density, \(|\Delta \mathbf{B}/B|\) and rotation angle). \(B_N/B\) and in-plane rotation angle, however, are significantly different.
\end{itemize}

\subsubsection{Future Work}\label{future-work-1}

\begin{itemize}
\tightlist
\item
  Understand the constant normalized thickness and current density of discontinuities with radial distance and change in the \(B_N/B\) and in-plane rotation angle.
\end{itemize}


\bibliography{../../../../files/bibliography/research.bib}


\bibliography{files/Anton.addon.bib,files/Anton.full.bib,files/research.bib}


\end{document}
