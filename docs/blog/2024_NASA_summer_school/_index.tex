% Options for packages loaded elsewhere
\PassOptionsToPackage{unicode}{hyperref}
\PassOptionsToPackage{hyphens}{url}
\PassOptionsToPackage{dvipsnames,svgnames,x11names}{xcolor}
%
\documentclass[
  letterpaper,
  DIV=11,
  numbers=noendperiod]{scrartcl}

\usepackage{amsmath,amssymb}
\usepackage{iftex}
\ifPDFTeX
  \usepackage[T1]{fontenc}
  \usepackage[utf8]{inputenc}
  \usepackage{textcomp} % provide euro and other symbols
\else % if luatex or xetex
  \usepackage{unicode-math}
  \defaultfontfeatures{Scale=MatchLowercase}
  \defaultfontfeatures[\rmfamily]{Ligatures=TeX,Scale=1}
\fi
\usepackage{lmodern}
\ifPDFTeX\else  
    % xetex/luatex font selection
\fi
% Use upquote if available, for straight quotes in verbatim environments
\IfFileExists{upquote.sty}{\usepackage{upquote}}{}
\IfFileExists{microtype.sty}{% use microtype if available
  \usepackage[]{microtype}
  \UseMicrotypeSet[protrusion]{basicmath} % disable protrusion for tt fonts
}{}
\makeatletter
\@ifundefined{KOMAClassName}{% if non-KOMA class
  \IfFileExists{parskip.sty}{%
    \usepackage{parskip}
  }{% else
    \setlength{\parindent}{0pt}
    \setlength{\parskip}{6pt plus 2pt minus 1pt}}
}{% if KOMA class
  \KOMAoptions{parskip=half}}
\makeatother
\usepackage{xcolor}
\ifLuaTeX
  \usepackage{luacolor}
  \usepackage[soul]{lua-ul}
\else
  \usepackage{soul}
  
\fi
\setlength{\emergencystretch}{3em} % prevent overfull lines
\setcounter{secnumdepth}{-\maxdimen} % remove section numbering
% Make \paragraph and \subparagraph free-standing
\ifx\paragraph\undefined\else
  \let\oldparagraph\paragraph
  \renewcommand{\paragraph}[1]{\oldparagraph{#1}\mbox{}}
\fi
\ifx\subparagraph\undefined\else
  \let\oldsubparagraph\subparagraph
  \renewcommand{\subparagraph}[1]{\oldsubparagraph{#1}\mbox{}}
\fi
\pagestyle{empty}


\providecommand{\tightlist}{%
  \setlength{\itemsep}{0pt}\setlength{\parskip}{0pt}}\usepackage{longtable,booktabs,array}
\usepackage{calc} % for calculating minipage widths
% Correct order of tables after \paragraph or \subparagraph
\usepackage{etoolbox}
\makeatletter
\patchcmd\longtable{\par}{\if@noskipsec\mbox{}\fi\par}{}{}
\makeatother
% Allow footnotes in longtable head/foot
\IfFileExists{footnotehyper.sty}{\usepackage{footnotehyper}}{\usepackage{footnote}}
\makesavenoteenv{longtable}
\usepackage{graphicx}
\makeatletter
\def\maxwidth{\ifdim\Gin@nat@width>\linewidth\linewidth\else\Gin@nat@width\fi}
\def\maxheight{\ifdim\Gin@nat@height>\textheight\textheight\else\Gin@nat@height\fi}
\makeatother
% Scale images if necessary, so that they will not overflow the page
% margins by default, and it is still possible to overwrite the defaults
% using explicit options in \includegraphics[width, height, ...]{}
\setkeys{Gin}{width=\maxwidth,height=\maxheight,keepaspectratio}
% Set default figure placement to htbp
\makeatletter
\def\fps@figure{htbp}
\makeatother

\addtokomafont{disposition}{\rmfamily} % Make the headings the same font as the rest of the document, instead of sans serif
\KOMAoption{captions}{tableheading}
\makeatletter
\@ifpackageloaded{tcolorbox}{}{\usepackage[skins,breakable]{tcolorbox}}
\@ifpackageloaded{fontawesome5}{}{\usepackage{fontawesome5}}
\definecolor{quarto-callout-color}{HTML}{909090}
\definecolor{quarto-callout-note-color}{HTML}{0758E5}
\definecolor{quarto-callout-important-color}{HTML}{CC1914}
\definecolor{quarto-callout-warning-color}{HTML}{EB9113}
\definecolor{quarto-callout-tip-color}{HTML}{00A047}
\definecolor{quarto-callout-caution-color}{HTML}{FC5300}
\definecolor{quarto-callout-color-frame}{HTML}{acacac}
\definecolor{quarto-callout-note-color-frame}{HTML}{4582ec}
\definecolor{quarto-callout-important-color-frame}{HTML}{d9534f}
\definecolor{quarto-callout-warning-color-frame}{HTML}{f0ad4e}
\definecolor{quarto-callout-tip-color-frame}{HTML}{02b875}
\definecolor{quarto-callout-caution-color-frame}{HTML}{fd7e14}
\makeatother
\makeatletter
\@ifpackageloaded{caption}{}{\usepackage{caption}}
\AtBeginDocument{%
\ifdefined\contentsname
  \renewcommand*\contentsname{Table of contents}
\else
  \newcommand\contentsname{Table of contents}
\fi
\ifdefined\listfigurename
  \renewcommand*\listfigurename{List of Figures}
\else
  \newcommand\listfigurename{List of Figures}
\fi
\ifdefined\listtablename
  \renewcommand*\listtablename{List of Tables}
\else
  \newcommand\listtablename{List of Tables}
\fi
\ifdefined\figurename
  \renewcommand*\figurename{Figure}
\else
  \newcommand\figurename{Figure}
\fi
\ifdefined\tablename
  \renewcommand*\tablename{Table}
\else
  \newcommand\tablename{Table}
\fi
}
\@ifpackageloaded{float}{}{\usepackage{float}}
\floatstyle{ruled}
\@ifundefined{c@chapter}{\newfloat{codelisting}{h}{lop}}{\newfloat{codelisting}{h}{lop}[chapter]}
\floatname{codelisting}{Listing}
\newcommand*\listoflistings{\listof{codelisting}{List of Listings}}
\makeatother
\makeatletter
\makeatother
\makeatletter
\@ifpackageloaded{caption}{}{\usepackage{caption}}
\@ifpackageloaded{subcaption}{}{\usepackage{subcaption}}
\makeatother
\ifLuaTeX
  \usepackage{selnolig}  % disable illegal ligatures
\fi
\usepackage{bookmark}

\IfFileExists{xurl.sty}{\usepackage{xurl}}{} % add URL line breaks if available
\urlstyle{same} % disable monospaced font for URLs
\hypersetup{
  pdftitle={NASA Heliophysics Summer School},
  colorlinks=true,
  linkcolor={blue},
  filecolor={Maroon},
  citecolor={Blue},
  urlcolor={Blue},
  pdfcreator={LaTeX via pandoc}}

\title{NASA Heliophysics Summer School}
\author{}
\date{}

\begin{document}
\maketitle

\newpage{}

\section{\texorpdfstring{\ul{Research
Statement}}{Research Statement}}\label{research-statement}

Zijin Zhang\\
Department of Earth, Planetary, and Space Sciences (EPSS)\\
University of California, Los Angeles (UCLA)

I am a second-year graduate student in the EPSS Department at UCLA, with
an interest in contributing to the understanding of key phenomena in the
heliosphere like magnetic discontinuities. Magnetic discontinuities,
characterized by localized, transient alterations in the magnetic field,
play a crucial role in in processes such as efficient \emph{plasma
acceleration} and the generation of \emph{plasma instabilities}
(magnetic reconnection) associated with discontinuity currents. To this
end, I leverage a multifaceted approach that encompasses
\href{https://beforerr.github.io/ids_spatial_evolution_juno/_manuscript/}{spacecraft
observations},
\href{https://beforerr.github.io/ion_scattering_by_SWD/}{numerical
simulations}, the development of
\href{https://beforerr.github.io/discontinuitypy/}{analytical tools} and
the analysis of extensive datasets to study these phenomena.

\subsection{Research experience}\label{research-experience}

My research journey has encompassed several key projects:

\begin{enumerate}
\def\labelenumi{\arabic{enumi}.}
\item
  analyzing the spatial evolution of solar wind discontinuities in the
  outer heliosphere with Juno spacecraft data by differentiating the
  temporal effect (correlated with solar activity) and spatial
  variations (correlated with radial distance);
\item
  integrating observations from multiple spacecraft missions (Van Allen
  Probes, ERG/ARASE, MMS, GOES, ELFIN, POES) to explore the dynamics of
  relativistic electrons in the magnetosphere with series of strong
  electron and ion injections from the plasma sheet and strong electron
  precipitation; and
\item
  employing particle-in-cell simulations to investigate the interaction
  between the solar wind and lunar crustal magnetic anomalies.
\end{enumerate}

\subsection{Future research plans}\label{future-research-plans}

Looking forward, my research will concentrate on elucidating the
influence of magnetic discontinuities on the acceleration of solar wind
particles. Utilizing data from the Parker Solar Probe (PSP) to examine
the spatial distribution of discontinuities in the inner heliosphere
represents a significant part of this work. By synthesizing PSP
observations with data from additional spacecraft, I aim to trace the
evolution of these discontinuities from their solar origins to the outer
reaches of the heliosphere. Subsequent numerical simulations,
particularly test-particle simulations incorporating realistic
discontinuity parameters, will enable a deeper understanding of solar
wind particle dynamics in the presence of discontinuities.

\subsection{Relevence of the school to my career
goals}\label{relevence-of-the-school-to-my-career-goals}

The thematic focus of the school on universal processes in heliophysics
aligns perfectly with my research interests, since discontinuities and
particle acceleration are ubiquitous in the heliosphere. Participating
in this program presents a unique chance to engage with the latest
advancements in the field and to network with leading experts and fellow
researchers. I am especially drawn to the \emph{comparative} dimension
of the school's curriculum, which promises to enrich my research
perspective by highlighting the commonalities and distinctions among
heliospheric processes across different regions. This comparative
analysis is invaluable, as it will broaden my comprehension of the
phenomena under study and enhance the contextual framework of my
research endeavors.

\newpage{}

\section{\texorpdfstring{\ul{Personal
Statement}}{Personal Statement}}\label{personal-statement}

Throughout my academic journey as a second-year graduate student in the
EPSS Department at UCLA, I have learned the value of integrating diverse
perspectives into scientific research. My focus on understanding
magnetic discontinuities within the heliosphere has not only deepened my
appreciation for the complexity of space phenomena but also underscored
the importance of collaborative approaches that embrace a wide range of
insights. This realization has been profoundly influenced by my personal
experiences and my interactions with a diverse group of researchers and
communities.

My research endeavors, from employing spacecraft observations and
numerical simulations to developing analytical tools, have always been
propelled by collaboration. In studying the spatial evolution of solar
wind discontinuities and the dynamics of relativistic electrons in the
magnetosphere, I have had the privilege of working with data from
missions like PSP, MMS, ELFIN, and ERG. Colloborating with other
researchers, especially with instrument teams, helped me to gain a
deeper understanding of the data (their limitations and capabilities)
and to develop new tools for data analysis. These experiences have
taught me that the richness of scientific discovery is greatly enhanced
by the diverse methodologies and perspectives that each team member
brings to the table.

Engagement with fellow researchers, especially in an interdisciplinary
field like heliophysics, demands openness to different viewpoints. The
heliosphere, host to complex phenomena like magnetic reconnection and
particle acceleration, encompasses a vast plasma parameter space, making
a holistic understanding contingent upon an approach that integrates
data analysis, numerical simulations, and theoretical modeling. By
actively seeking to include voices from various scientific backgrounds,
I have learned that the most innovative and transformative ideas often
emerge from the intersection of diverse perspectives. This approach not
only fosters a culture of mutual respect and learning but also propels
our research forward in unexpected and innovative directions.

My commitment to inclusivity in scientific conversations is also
informed by my personal experiences. The challenging winter
mountaineering expedition to Mount Siguniang and the fifteen-day field
research in the desertification Mu-Us Sandy Land have been pivotal in
shaping my understanding of resilience, perseverance, and the power of
diverse perspectives. These experiences taught me that whether facing
the harsh realities of nature or addressing global challenges like
climate change, collective efforts rooted in diverse experiences and
expertise can lead to meaningful change.

As I look forward to participating in the school focused on universal
processes in heliophysics, I am particularly excited about the
opportunity to interact with experts and peers from different regions
and research backgrounds. The comparative aspect of the school aligns
perfectly with my belief in the importance of diverse perspectives. I am
eager to learn about the similarities and differences in heliospheric
processes, which will undoubtedly enrich my research and contribute to a
deeper collective understanding of the universe we study.



\end{document}
